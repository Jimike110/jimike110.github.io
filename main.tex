%%%%%%%%%%%%%%%%%%%%%%%%%%%%%%%%%%%%%%%%%%%%%%%%%%%%%%%%%%%%%%%%%%%%%%%%%%%%%%%%
%
%   A "Killer" Resume Template for Michael Oladoye
%   Optimized for Russian Big Tech (Yandex, Avito, TBank)
%   Structure inspired by HeadHunter guide and best practices.
%
%%%%%%%%%%%%%%%%%%%%%%%%%%%%%%%%%%%%%%%%%%%%%%%%%%%%%%%%%%%%%%%%%%%%%%%%%%%%%%%%

\documentclass[a4paper,10pt]{article}

% ----- PACKAGES -----
\usepackage[T2A]{fontenc}
\usepackage[utf8]{inputenc}
\usepackage[russian]{babel}
\usepackage{geometry}
\usepackage{hyperref}
\usepackage{enumitem}
\usepackage{titlesec}
\usepackage{array} % For better column definitions in tables

% ----- PAGE & FONT SETUP -----
% Set margins
\geometry{top=1.2cm, bottom=1.5cm, left=1.5cm, right=1.5cm}
% Remove paragraph indentation
\setlength{\parindent}{0pt}

% ----- HYPERLINK SETUP -----
% Make links clickable and professional-looking
\hypersetup{
    colorlinks=true,
    urlcolor=blue,
    linkcolor=black,
    pdftitle={Резюме - Майкл Оладое},
    pdfauthor={Майкл Оладое}
}

% ----- SECTION FORMATTING -----
% Creates a horizontal line under each section title
\titleformat{\section}{\large\bfseries}{}{0em}{}[\titlerule\vspace{0.5ex}]
\titlespacing*{\section}{0pt}{2ex}{1.5ex} % Adjusts space around section titles

% ----- CUSTOM COMMANDS -----
% For job/education entries to keep formatting consistent
\newcommand{\entry}[3]{%
  \noindent\textbf{#1} | #2 \hfill \textbf{\textit{#3}} \par
}


\begin{document}

% =============================================================================
% HEADER
% =============================================================================
\begin{center}
    % --- Name ---
    {\Huge \textbf{Майкл Оладое}} \\[5pt]
    % --- Title ---
    {\large Frontend Developer (React.js / Next.js / TypeScript)} \\[6pt]
    % --- Contact Info ---
    \href{tel:+79967866779}{+7 996 786 67-79} \quad|\quad
    \href{mailto:m.oladoye@yandex.com}{m.oladoye@yandex.com} \quad|\quad
    \href{https://jimike.vercel.app}{Портфолио: jimike.vercel.app} \\[4pt]
    \href{https://github.com/Jimike110}{GitHub: Jimike110} \quad|\quad
    \href{https://www.linkedin.com/in/jimike}{LinkedIn: jimike} \quad|\quad
    Санкт-Петербург, Россия
\end{center}


% =============================================================================
% SUMMARY
% =============================================================================
\section*{Обо мне}
Проактивный Frontend-разработчик с 3+ летним опытом создания сложных и производительных веб-приложений на React и TypeScript. Специализируюсь на разработке интуитивно понятных пользовательских интерфейсов и оптимизации Core Web Vitals. Финалист хакатонов от Яндекса и Сбера. Ищу позицию Frontend-разработчика для участия в создании инновационных продуктов и развития в команде профессионалов.


% =============================================================================
% SKILLS
% =============================================================================
\section*{Ключевые компетенции}
\begin{tabular}{ >{\bfseries}l @{\hspace{1em}} l }
    Frontend: & React.js, Next.js, TypeScript, JavaScript, Redux (Toolkit), HTML5, CSS3, Tailwind CSS \\
    Backend \& DB: & Node.js, RESTful API, PostgreSQL, Supabase \\
    Инструменты: & Git, GitHub, CI/CD, Docker, Jira, Figma, Notion, Vite \\
    Тестирование: & Jest, React Testing Library \\
    Языки: & Английский (C2, свободное владение), Русский (B2, уверенное владение) \\
\end{tabular}


% =============================================================================
% EXPERIENCE
% =============================================================================
\section*{Опыт работы}
\entry{Freelance Frontend Developer}{Удаленно}{Март 2024 – Настоящее время}
\begin{itemize}[leftmargin=*, topsep=0.5ex, itemsep=0.2ex]
    \item Разработал и запустил с нуля e-commerce платформу для fashion-бренда Zuccini Studios, что привело к \textbf{росту онлайн-продаж на 40\%} и увеличению конверсии на 15\% в первые 3 месяца.
    \item Возглавил техническую реанимацию платформы cdaem, восстановив её из нерабочего состояния, что привело к привлечению \textbf{более 1,500 активных пользователей} и публикации \textbf{5,000+ новых объявлений за 3 месяца}. Внедрил real-time чат и новые личные кабинеты, а также устранил \textbf{100\% критических ошибок} в системе оплаты, возобновив коммерческую деятельность проекта.
    \item Внедрил адаптивную верстку для нескольких клиентских сайтов, обеспечив 100\% корректное отображение на мобильных устройствах и \textbf{увеличив мобильный трафик на 30\%}.
\end{itemize}
\vspace{1ex}

\entry{Стажёр-разработчик Frontend}{SkailleUp (Нью-Йорк), Удаленно}{Август 2023 – Март 2024}
\begin{itemize}[leftmargin=*, topsep=0.5ex, itemsep=0.2ex]
    \item Участвовал в миграции легаси-кода на современный стек (React, TypeScript), что \textbf{повысило стабильность приложения и ускорило разработку новых функций на 20\%}.
    \item Разработал более 15 переиспользуемых UI-компонентов, что \textbf{сократило дублирование кода на 30\%} в проекте.
    \item Активно участвовал в code review, помогая поддерживать высокие стандарты качества кода и командной работы.
\end{itemize}
\vspace{1ex}

\entry{ИТ-ассистент}{D’Lens Digital Technologies}{Май 2022 – Июль 2023}
\begin{itemize}[leftmargin=*, topsep=0.5ex, itemsep=0.2ex]
    \item Координировал и автоматизировал процессы обновления ПО и оборудования, \textbf{повысив операционную эффективность компании на 30\%}.
\end{itemize}


% =============================================================================
% PROJECTS
% =============================================================================
\section*{Проекты}
\noindent\textbf{Платформа бронирования туров на Камчатке} \hfill \href{https://kamchatka-adventures.vercel.app/}{\textbf{[Демо]}} \quad \href{https://github.com/Jimike110/Kamchatka/}{\textbf{[Код]}} \\
\begin{itemize}[leftmargin=*, topsep=0.5ex, itemsep=0.2ex]
    \item \textbf{Описание:} Full-stack платформа корпоративного уровня для бронирования туров. Реализована система управления бронированиями в реальном времени, мультивалютные платежи и ролевая модель доступа.
    \item \textbf{Стек:} React 18, TypeScript, Tailwind CSS, Supabase, PostgreSQL, Edge Functions.
\end{itemize}
\vspace{1ex}

\noindent\textbf{Bankify — онлайн-банк} \hfill \href{https://bankify-jimike.vercel.app/}{\textbf{[Демо]}} \quad \href{https://github.com/Jimike110/bankify}{\textbf{[Код]}} \\
\begin{itemize}[leftmargin=*, topsep=0.5ex, itemsep=0.2ex]
    \item \textbf{Описание:} Прототип онлайн-банка с современным интерфейсом для управления счетами и просмотра транзакций. Интегрированы интерактивные дашборды с использованием Chart.js.
    \item \textbf{Стек:} Next.js, React, TypeScript, Tailwind CSS, Chart.js.
\end{itemize}


% =============================================================================
% ACHIEVEMENTS
% =============================================================================
\section*{Достижения}
\begin{itemize}[leftmargin=*, topsep=0.5ex, itemsep=0.2ex]
    \item \href{https://disk.yandex.ru/i/wE10Gv0LYPanjg}{\textbf{Финалист} хакатона «Сбер х Просто х ИТМО» (2025)}
    \item \href{https://certify.s3.yandex.net/young-yandex/74b6021d-3100-422c-b4d4-0a0d8d1833ea/4df33b64-d63d-402a-92cb-dd5955e80ff3.pdf}{\textbf{Полуфиналист} хакатона «Баттл Вузов. Кубок Y\&Y», (Яндекс, 2024)}
\end{itemize}


% =============================================================================
% EDUCATION
% =============================================================================
\section*{Образование}
\entry{Университет ИТМО}{Санкт-Петербург}{2024 – 2028}
\textit{Бакалавриат, Факультет Программной Инженерии и Компьютерной Техники}

\end{document}
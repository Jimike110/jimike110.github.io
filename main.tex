%%%%%%%%%%%%%%%%%%%%%%%%%%%%%%%%%%%%%%%%%%%%%%%%%%%%%%%%%%%%%%%%%%%%%%%%%%%%%%%%
%
%   Killer Resume — Michael Oladoye
%   Optimized for Russian Big Tech (Yandex, Avito, TBank)
%   ATS-friendly + SEO-optimized
%
%%%%%%%%%%%%%%%%%%%%%%%%%%%%%%%%%%%%%%%%%%%%%%%%%%%%%%%%%%%%%%%%%%%%%%%%%%%%%%%%

\documentclass[a4paper,11pt]{article}

% ----- PACKAGES -----
\usepackage[T2A]{fontenc}
\usepackage[utf8]{inputenc}
\usepackage[russian]{babel}
\usepackage{geometry}
\usepackage{hyperref}
\usepackage{enumitem}
\usepackage{titlesec}
\usepackage{array}

% ----- PAGE & FONT SETUP -----
\geometry{top=1.2cm, bottom=1.5cm, left=1.5cm, right=1.5cm}
\setlength{\parindent}{0pt}

% ----- HYPERLINK SETUP -----
\hypersetup{
    colorlinks=true,
    urlcolor=blue,
    linkcolor=black,
    pdftitle={Резюме - Майкл Оладое},
    pdfauthor={Майкл Оладое}
}

% ----- SECTION FORMATTING -----
\titleformat{\section}{\large\bfseries}{}{0em}{}[\titlerule\vspace{0.5ex}]
\titlespacing*{\section}{0pt}{2ex}{1.5ex}

% ----- CUSTOM COMMANDS -----
\newcommand{\entry}[3]{%
  \noindent\textbf{#1} | #2 \hfill \textbf{\textit{#3}} \par
}

\begin{document}

% =============================================================================
% HEADER
% =============================================================================
\begin{center}
    {\Huge \textbf{Майкл Оладое}} \\[5pt]
    {\large Frontend Developer (React.js / Next.js / TypeScript)} \\[6pt]
    Telegram: \href {https://t.me/jimikeCodes}{@jimikeCodes} \quad|\quad
    \href{tel:+79967866779}{+7 996 786 67-79} \quad|\quad
    \href{mailto:m.oladoye@yandex.com}{m.oladoye@yandex.com} \quad|\quad
    \href{https://jimike.vercel.app}{Портфолио: jimike.vercel.app} \\[4pt]
    \href{https://github.com/Jimike110}{GitHub: Jimike110} \quad|\quad
    \href{https://www.linkedin.com/in/jimike}{LinkedIn: jimike} \quad|\quad
    Санкт-Петербург, Россия
\end{center}


% =============================================================================
% SUMMARY
% =============================================================================
\section*{Обо мне}
Проактивный Frontend-разработчик (React / Next.js / TypeScript) с более чем 3 годами коммерческого опыта в создании современных, производительных и масштабируемых веб-приложений.  
Специализируюсь на разработке интерфейсов (UI/UX), оптимизации Core Web Vitals и проектировании архитектуры клиентской части под реальные продуктовые задачи.  

Создаю проекты полного цикла — от архитектуры и проектирования компонентов до продакшн-сборки с CI/CD.  
Владею стеком React 18, Next.js (SSR, SSG), TypeScript, Redux Toolkit, REST API, Tailwind CSS. Соблюдаю принципы чистого кода, инженерного мышления и доступности (Accessibility, A11y).  

Опыт работы в международных командах (США, Россия) позволил мне развить гибкость, коммуникацию и лидерские качества.  

Финалист и полуфиналист хакатонов от Яндекса и Сбера.  
Разрабатывал e-commerce, финтех и веб-сервисы полного цикла.  
Оптимизировал производительность до 90+ Core Web Vitals по Google Lighthouse.  

\textbf{Основной стек:} React.js, Next.js, TypeScript, JavaScript (ES6+), Redux Toolkit, Tailwind CSS, REST API, Node.js, PostgreSQL, Supabase, Docker, CI/CD, Git, Figma.


% =============================================================================
% EXPERIENCE
% =============================================================================
\section*{Опыт работы}

\entry{Frontend Developer}{Jimike Codes}{Март 2024 – Настоящее время}
Реализую комплексную разработку и поддержку высоконагруженных веб-приложений полного цикла на React.js, Next.js, TypeScript, JavaScript, HTML5, CSS3 и Tailwind CSS. Отвечаю за архитектуру клиентской части, оптимизацию производительности, доступность (A11y) и качество кода.

\textbf{Ключевые достижения:}
\begin{itemize}[leftmargin=*, topsep=0.5ex, itemsep=0.2ex]
    \item Разработал и запустил с нуля e-commerce платформу для fashion-бренда \textbf{Zuccini Studios} (\textit{React + Next.js + TypeScript + Node.js + PostgreSQL}), обеспечив стабильную работу и адаптивную верстку.  
    → \textbf{Результат:} рост онлайн-продаж на 40\%, увеличение конверсии на 15\%.
    \item Возглавил техническую реинкарнацию платформы \textbf{cdaem.ru}, восстановив и модернизировав устаревший код.  
    → Внедрил SSR и SPA-архитектуру, повысив скорость загрузки на 60\% и Core Web Vitals до 90+ баллов.  
    → Организовал работу команды из 3 разработчиков, внедрил CI/CD через GitHub Actions.  
    → Восстановил систему оплаты и внедрил real-time чат (WebSocket), личные кабинеты, REST API.  
    → Платформа обслуживает \textbf{1500+ активных пользователей и 7300+ объявлений}.
    \item Внедрил адаптивную и доступную верстку (responsive + accessible UI) на React и Tailwind CSS.  
    → Рост мобильного трафика на 30\%.
    \item Провел оптимизацию производительности (lazy-loading, code splitting, кеширование, image optimization), сократив TTFB и LCP на 45\%.
\end{itemize}
\vspace{1ex}

\entry{Стажёр Frontend-разработчик}{SkailleUp (Нью-Йорк, удалённо)}{Август 2023 – Март 2024}
\begin{itemize}[leftmargin=*, topsep=0.5ex, itemsep=0.2ex]
    \item Участвовал в миграции легаси-кода на React + TypeScript, повысив стабильность приложения и ускорив выпуск новых фич на 20\%.
    \item Разработал 15+ UI-компонентов, сократив дублирование кода на 30\%.
    \item Участвовал в code review, поддерживая высокий уровень инженерного качества.
\end{itemize}
\vspace{1ex}

\entry{Системный администратор}{D’Lens Digital Technologies}{Май 2022 – Июль 2023}
\begin{itemize}[leftmargin=*, topsep=0.5ex, itemsep=0.2ex]
    \item Координировал обновление ПО и оборудования, автоматизировав внутренние процессы.  
    → Повышение операционной эффективности компании на 30\%.
\end{itemize}


% =============================================================================
% PROJECTS
% =============================================================================
\section*{Проекты}
\noindent\textbf{Kamchatka Adventures — платформа бронирования туров} \hfill \href{https://kamchatka-adventures.vercel.app/}{[Демо]} \quad \href{https://github.com/Jimike110/Kamchatka/}{[Код]} \\
\begin{itemize}[leftmargin=*, topsep=0.5ex, itemsep=0.2ex]
    \item Full-stack приложение корпоративного уровня для бронирования туров. Реализована система управления бронированиями, мультивалютные платежи и ролевая модель доступа.
    \item \textbf{Стек:} React 18, TypeScript, Tailwind CSS, Supabase, PostgreSQL, Edge Functions.
\end{itemize}
\vspace{1ex}

\noindent\textbf{Bankify — онлайн-банк} \hfill \href{https://bankify-jimike.vercel.app/}{[Демо]} \quad \href{https://github.com/Jimike110/bankify}{[Код]} \\
\begin{itemize}[leftmargin=*, topsep=0.5ex, itemsep=0.2ex]
    \item Прототип онлайн-банка с современным UI для управления счетами и транзакциями.  
    Интерактивные дашборды с Chart.js и адаптивный интерфейс.
    \item \textbf{Стек:} Next.js, React, TypeScript, Tailwind CSS, Chart.js.
\end{itemize}


% =============================================================================
% ACHIEVEMENTS
% =============================================================================
\section*{Достижения}
\begin{itemize}[leftmargin=*, topsep=0.5ex, itemsep=0.2ex]
    \item \href{https://disk.yandex.ru/i/wE10Gv0LYPanjg}{\textbf{Финалист}} хакатона «Сбер × Просто × ИТМО» (2025)
    \item \href{https://certify.s3.yandex.net/young-yandex/74b6021d-3100-422c-b4d4-0a0d8d1833ea/4df33b64-d63d-402a-92cb-dd5955e80ff3.pdf}{\textbf{Полуфиналист}} хакатона «Баттл Вузов. Кубок Y\&Y» (Яндекс, 2024)
\end{itemize}


% =============================================================================
% EDUCATION
% =============================================================================
\section*{Образование}
\entry{Университет ИТМО}{Санкт-Петербург}{2024 – 2028}
\textit{Бакалавриат, Программная инженерия и компьютерные технологии}

\end{document}
